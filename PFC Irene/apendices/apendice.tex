%\appendix
%\clearpage
%\addappheadtoctoc
%\appendixpage
\chapter{Descomposici�n de matrices}

Se denomina \textit{autovalor} de la matriz cuadrada $\miA$ al valor escalar $\lambda$ para el cual existe un vector $\mix$ distinto de $\mathbf{0}$ tal que $\miA \mix = \lambda \mix$. Al vector $\mix$ se le llama \textit{autovector} de $\miA$ correspondiente a $\lambda$. Los autovalores de la matriz $\miA$ son aquellos que valores de $\lambda$ que satisfacen la \textit{ecuaci�n caracter�stica} de $\miA$, que se define como det[$\miA$ - $\lambda \mathbf{I}$] = 0. El polinomio en $\lambda$ definido por det[$\miA$ - $\lambda \mathbf{I}$] = 0 se llama \textit{polinomio caracter�stico} de $\miA$, por lo que los autovalores de $\miA$ son las ra�ces del polinomio caracter�stico. El polinomio caracter�stico de una matriz $N \times N$ tiene $N$ ra�ces �nicas $r_1, ..., r_N (r_i \neq r_j)$ si es de la forma  det[$\miA - \lambda \mathbf{I}$] = $(-1)^N(\lambda - r_1)...(\lambda - r_N)$. Cuando el polinomio caracter�stico incluye un t�rmino $(\lambda - r_i)^k, k>1$, decimos que la ra�z $r_i$ tiene multiplicidad $k$. Una matriz $N \times N$ tiene $N$ autovalores $\lambda_1, ... \lambda_N$, aunque no todos ellos tienen por qu� ser �nicos si alguna de las ra�ces tiene multiplicidad mayor que 1. Adem�s, el determinante de una matriz es el producto de todos los autovalores de la matriz
\footnote{
Esta propiedad es f�cilmente demostrable de la siguiente forma:
Suponiendo que $\lambda_1, ... \lambda_n$ son los autovalores de la matriz $\miA$. As�, como ya hemos dicho, los autovalores son tambi�n las ra�ces del polinomio caracter�stico
\begin{equation}
	det(\miA - \lambda \mathbf{I}) = p(\lambda) = (-1)^n(\lambda - \lambda_1)...(\lambda - \lambda_n) = (\lambda_1 - \lambda)...(\lambda_n - \lambda)
\end{equation}
As�, dando a $\lambda$ el valor 0, simplemente porque es una variable, 
\begin{equation}
	det(\miA) = \lambda_1 \lambda_2... \lambda_n
\end{equation}
es decir, el determinante de la matriz es igual al producto de sus autovalores}, 

teniendo en cuenta que un autovalor $r_i$ con multiplicidad $k$ contribuir� $r_i^k$ al producto. 

Los autovalores de una matriz herm�tica son siempre reales, aunque los autovectores pueden ser complejos. Adem�s, si $\miA$ es una matriz $N \times N$ se puede escribir como sigue:
\begin{equation}
	\miA = \mathbf{U} \mathbf{\Lambda} \mathbf{U}^H
\end{equation}
donde $\mathbf{U}$ es una matriz unitaria cuyas columnas son los autovectores de $\miA$ y $\mathbf{\Lambda}$ = diag[$\lambda_1$, ..., $\lambda_k$, 0, ..., 0]. Como ya hemos comentado, cuando $\miA$ es herm�tica, $\mathbf{\Lambda}$ tiene s�lo elementos reales. Decimos que la matriz  $\miA$ es \textit{definida positiva} si, para todos los vectores $\mix \neq 0$, tenemos que $\mix^H\miA\mix > 0$. Una matriz herm�tica es definida positiva si y s�lo si todos sus autovalores son positivos. De forma similar, la matriz $\miA$ es semidefinida positiva o definida no negativa si, para todos los vectores $\mix \neq 0$, $\miX^H\miA\mix \ge 0$. Una matriz herm�tica es definida no negativa si y s�lo si todos sus autovalores son no negativos. 

Suponiendo que la matriz $\miA$ es una matriz de dimensiones $N \times M$ y tiene rango $R_A$. Entonces existe una matriz $\mathbf{\Sigma}$ de dimensiones $N$x$M$, y dos matrices unitarias $\mathbf{U}$ y $\mathbf{V}$ cuadradas de tama�o $N \times M$ respectivamente tal que
\begin{equation}\label{svd}
\miA = \mathbf{U}\mathbf{\Sigma}\mathbf{V}^H
\end{equation}
Llamamos a las columnas de $\mathbf{V}$ los vectores singulares de $\miA$ por la derecha y las columnas de $\mathbf{U}$ los vectores singulares de $\miA$ por la izquierda. En cuanto a la matriz $\mathbf{\Sigma}$, es una matriz en cuya diagonal se encuentran los valores singulares de $\miA$ y todos los elementos fuera de la diagonal son cero

\begin{equation}
\mathbf{\Sigma}_{N \times M}=
\begin{bmatrix}
\sigma_1 & \cdots & 0\\
\vdots & \ddots & \vdots \\
0 & \cdots & \sigma_N
\end{bmatrix}\\ 
\end{equation}
para N = M. Para los casos en los que la matriz no es cuadrada:
\begin{equation}
\mathbf{\Sigma}_{N \times M}=
\begin{bmatrix}
\sigma_1 & \cdots & 0\\
\vdots & \ddots & \vdots \\
0 & \cdots & \sigma_N\\
0 & \cdots & 0\\
\vdots & \ddots & \vdots \\
0 & \cdots & 0\\
\end{bmatrix}\\ 
\end{equation}
para $N > M$, y
\begin{equation}
\mathbf{\Sigma}_{N \times M}=
\begin{bmatrix}
\sigma_1 & \cdots & 0 & 0 & \cdots & 0\\
\vdots & \ddots & \vdots  & \vdots &\ddots & \vdots\\
0 & \cdots & \sigma_N & 0 & \cdots & 0
\end{bmatrix}\\ 
\end{equation}
para $N < M$, donde $\sigma_i = \sqrt{\lambda_i}$ siendo $\sigma_i$ el valor singular $i$-�simo de la matriz $\miA\miA^H$ y $\lambda_i$ el autovalor $i$-�simo. A la descomposici�n descrita en (\ref{svd}) se le llama descomposici�n en valores singulares de $\miA$.







